%-----------------------------------------------------------------------------------------------------------------------------------------------%
%	The MIT License (MIT)
%
%	Copyright (c) 2021 Jitin Nair
%
%	Permission is hereby granted, free of charge, to any person obtaining a copy
%	of this software and associated documentation files (the "Software"), to deal
%	in the Software without restriction, including without limitation the rights
%	to use, copy, modify, merge, publish, distribute, sublicense, and/or sell
%	copies of the Software, and to permit persons to whom the Software is
%	furnished to do so, subject to the following conditions:
%	
%	THE SOFTWARE IS PROVIDED "AS IS", WITHOUT WARRANTY OF ANY KIND, EXPRESS OR
%	IMPLIED, INCLUDING BUT NOT LIMITED TO THE WARRANTIES OF MERCHANTABILITY,
%	FITNESS FOR A PARTICULAR PURPOSE AND NONINFRINGEMENT. IN NO EVENT SHALL THE
%	AUTHORS OR COPYRIGHT HOLDERS BE LIABLE FOR ANY CLAIM, DAMAGES OR OTHER
%	LIABILITY, WHETHER IN AN ACTION OF CONTRACT, TORT OR OTHERWISE, ARISING FROM,
%	OUT OF OR IN CONNECTION WITH THE SOFTWARE OR THE USE OR OTHER DEALINGS IN
%	THE SOFTWARE.
%	
%
%-----------------------------------------------------------------------------------------------------------------------------------------------%

%----------------------------------------------------------------------------------------
%	DOCUMENT DEFINITION
%----------------------------------------------------------------------------------------

% article class because we want to fully customize the page and not use a cv template
\documentclass[a4paper,12pt]{article}

%----------------------------------------------------------------------------------------
%	FONT
%----------------------------------------------------------------------------------------

% % fontspec allows you to use TTF/OTF fonts directly
% \usepackage{fontspec}
% \defaultfontfeatures{Ligatures=TeX}

% % modified for ShareLaTeX use
% \setmainfont[
% SmallCapsFont = Fontin-SmallCaps.otf,
% BoldFont = Fontin-Bold.otf,
% ItalicFont = Fontin-Italic.otf
% ]
% {Fontin.otf}

%----------------------------------------------------------------------------------------
%	PACKAGES
%----------------------------------------------------------------------------------------
\usepackage{url}
\usepackage{parskip} 	

%other packages for formatting
\RequirePackage{color}
\RequirePackage{graphicx}
\usepackage[usenames,dvipsnames]{xcolor}
\usepackage[scale=0.9]{geometry}

%tabularx environment
\usepackage{tabularx}

%for lists within experience section
\usepackage{enumitem}

% centered version of 'X' col. type
\newcolumntype{C}{>{\centering\arraybackslash}X} 

%to prevent spillover of tabular into next pages
\usepackage{supertabular}
\usepackage{tabularx}
\newlength{\fullcollw}
\setlength{\fullcollw}{0.47\textwidth}

%custom \section
\usepackage{titlesec}				
\usepackage{multicol}
\usepackage{multirow}

%CV Sections inspired by: 
%http://stefano.italians.nl/archives/26
\titleformat{\section}{\Large\scshape\raggedright}{}{0em}{}[\titlerule]
\titlespacing{\section}{0pt}{10pt}{10pt}

%for publications
\usepackage[style=authoryear,sorting=ydnt, maxbibnames=2]{biblatex}

%Setup hyperref package, and colours for links
\usepackage[unicode, draft=false]{hyperref}
\definecolor{linkcolour}{rgb}{0,0.2,0.6}
\hypersetup{colorlinks,breaklinks,urlcolor=linkcolour,linkcolor=linkcolour}
\addbibresource{citations.bib}
\setlength\bibitemsep{1em}

%for social icons
\usepackage{fontawesome5}

%debug page outer frames
%\usepackage{showframe}

%----------------------------------------------------------------------------------------
%	BEGIN DOCUMENT
%----------------------------------------------------------------------------------------
\begin{document}

% non-numbered pages
\pagestyle{empty} 

%----------------------------------------------------------------------------------------
%	TITLE
%----------------------------------------------------------------------------------------

% \begin{tabularx}{\linewidth}{ @{}X X@{} }
% \huge{Your Name}\vspace{2pt} & \hfill \emoji{incoming-envelope} email@email.com \\
% \raisebox{-0.05\height}\faGithub\ username \ | \
% \raisebox{-0.00\height}\faLinkedin\ username \ | \ \raisebox{-0.05\height}\faGlobe \ mysite.com  & \hfill \emoji{calling} number
% \end{tabularx}

\begin{tabularx}{\linewidth}{@{} C @{}}
\Huge{Anja Conev} \\[7.5pt]
\href{https://github.com/AnjaConev}{\raisebox{-0.05\height}\faGithub\ AnjaConev} \ $|$ \ 
%\href{https://linkedin.com/in/username}{\raisebox{-0.05\height}\faLinkedin\ username} \ $|$ \ 
%\href{https://mysite.com}{\raisebox{-0.05\height}\faGlobe \ mysite.com} \ $|$ \ 
\href{mailto:ac121@rice.edu}{\raisebox{-0.05\height}\faEnvelope \ ac121@rice.edu} \ $|$ \ 
\href{https://scholar.google.com/citations?user=NYWrBvYAAAAJ}{\raisebox{-0.05\height}\faGraduationCap\ Google Scholar}
%\href{tel:+000000000000}{\raisebox{-0.05\height}\faMobile \ +00.00.000.000} \\
\end{tabularx}

%----------------------------------------------------------------------------------------
% EXPERIENCE SECTIONS
%----------------------------------------------------------------------------------------

%Interests/ Keywords/ Summary
\section{Research interest}

I am broadly interested in \textbf{applied artificial intelligence and machine learning} (ML) in the field of \textbf{computational structural biology} with the goal of aiding drug discovery. In particular, the computational problems I explored include: design of ML-driven molecular scoring functions, search approaches in molecular docking, analysis and visualization of the data generated by \textbf{biomolecular and macromolecular simulations}. %I am currently exploring the  the potential use of ML models for ensemble docking of large ligands.

%On the computational side, my focus is in adapting algorithms to fit the unique challenges of the structural biological data. In particular the computational problems I explored include: design of ML-driven molecular scoring functions, search approaches in molecular docking, analysis and visualization of the data generated by molecular dynamics simulations. I am currently exploring the  the potential use of ML models for ensemble docking of large ligands. \\[3.75pt]
%On the application side, my previous work has been tightly coupled to computational modeling of proteins related to the human immune system (see projects\textit{ 3pHLA, SARS-Arena, HLA-Arena}). However, I am currently exploring a wider range of applications of our tools to cancer-related proteins.

%Experience


%----------------------------------------------------------------------------------------
%	EDUCATION
%----------------------------------------------------------------------------------------
\section{Education}
\begin{tabularx}{\linewidth}{@{}l X@{}}	

2019 - present & PhD in Computer Science \hfill (GPA: 4.0/4.0) \\ 
& advisor: Dr. Lydia Kavraki \\
& \textbf{Rice University} \\
& Houston, TX \\
\\
2014 - 2019 & B.S. in Electrical Engineering and Computer Technology \hfill  (GPA: 9.1/10) \\
& \textbf{University of Belgrade,} \\
& Belgrade, Serbia \\
\end{tabularx}


%----------------------------------------------------------------------------------------
%	TRAINING
%----------------------------------------------------------------------------------------

\section{Training}

\begin{tabularx}{\linewidth}{ @{}l r@{} }
\textbf{ITCR Trainee Workshop} & \hfill Sept 2022 \\
\textit{Washington University, School of Medicine, St Louis} & \\[3.75pt]
\multicolumn{2}{@{}X@{}}{
\begin{minipage}[t]{\linewidth}
    \begin{itemize}[nosep,after=\strut, leftmargin=1em, itemsep=3pt]
        \item[--] Workshop at the National Cancer Institute's annual conference in Information Technologies for Cancer Research (ITCR)
        \item[--] Acquired more advanced github skills and collaborated with multidisciplinary colleagues who work on developing computational tools for cancer research
    \end{itemize}
    \end{minipage}
}
\end{tabularx}


\begin{tabularx}{\linewidth}{ @{}l r@{} }
\textbf{Computer Science Student Advancement Program} & \hfill July - Oct 2018 \\
\textit{Kavraki Lab, Rice University, Houston, Texas} & \\[3.75pt]
\multicolumn{2}{@{}X@{}}{
\begin{minipage}[t]{\linewidth}
    \begin{itemize}[nosep,after=\strut, leftmargin=1em, itemsep=3pt]
        \item[--] Worked under the mentorship of Dr. Lydia Kavraki
        \item[--] Acquired experience in conducting scientific research while working in an interdisciplinary field of molecular docking.
        \item[--] Conducted a project – “Evaluation of the scoring functions' ranking power for peptide MHC-I complexes”
    \end{itemize}
    \end{minipage}
}
\end{tabularx}

\begin{tabularx}{\linewidth}{ @{}l r@{} }
\textbf{Summer school of machine learning} & \hfill July 2017 \\
\textit{Petnica Science Center and Microsoft Development Center, Serbia} & \\[3.75pt]
\multicolumn{2}{@{}X@{}}{
\begin{minipage}[t]{\linewidth}
    \begin{itemize}[nosep,after=\strut, leftmargin=1em, itemsep=3pt]
        \item[--] Acquired the theoretical and practical knowledge on state of the art programming methods in the field of machine learning.
        \item[--] Worked on a project – “Discriminative voxel modelling with convolutional neural networks”
    \end{itemize}
    \end{minipage}
}
\end{tabularx}

\begin{tabularx}{\linewidth}{ @{}l r@{} }
\textbf{Experimental Chemistry, Biology and Biomedicine research camps} & \hfill 2011 - 2013 \\
\textit{Petnica Science Center, Serbia} & \\[3.75pt]
\multicolumn{2}{@{}X@{}}{
\begin{minipage}[t]{\linewidth}
    \begin{itemize}[nosep,after=\strut, leftmargin=1em, itemsep=3pt]
        \item[--] Gained deeper theoretical knowledge base in the fields of biology, chemistry and biomedicine.
        \item[--]Learned the basics of conducting research and writing a scientific paper.
        \item[--] Worked on two projects ("The Effect of Lead on Fitness Components in D. sub.”,  "Stage and Sex Structure of Red-Backed Shrike Nesting Groups on the Territory of Petnica")
    \end{itemize}
    \end{minipage}
}
\end{tabularx}

%----------------------------------------------------------------------------------------
%	TEACHING
%----------------------------------------------------------------------------------------

\section{Teaching/Mentorship}

\begin{tabularx}{\linewidth}{ @{}l r@{} }
\textbf{Teaching assistant at Rice University} & \hfill 2020-2021 \\
\textit{Computer Science Department, Rice University} & \\

\multicolumn{2}{@{}X@{}}{
\begin{minipage}[t]{\linewidth}
Experience running recitation sessions, constructing quiz questions, grading, running office hours, writing \href{https://pacwar2020.blogs.rice.edu/2020/10/08/dont-walk-into-a-dead-end-street/}{blogs} for the final project.  \end{minipage}
}\\ \\
\multicolumn{2}{@{}X@{}}{
\begin{minipage}[t]{\linewidth}
\begin{itemize}[nosep,after=\strut, leftmargin=1em, itemsep=3pt]
        \item[--] COMP 557 (Artificial Intelligence) - Fall 2021
        \item[--] COMP 540 (Statistical Machine Learning) - Spring 2021
        \item[--] COMP 557 (Artificial Intelligence) - Fall 2020
    \end{itemize}
\end{minipage}
}
\end{tabularx}

\begin{tabularx}{\linewidth}{ @{}l r@{} }
\textbf{Co-mentoring interning students at KavrakiLab} & \hfill 2021-2022 \\
\textit{\href{https://kavrakilab.org/}{KavrakiLab}, Computer Science Department, Rice University} & \\

\multicolumn{2}{@{}X@{}}{
\begin{minipage}[t]{\linewidth}
Experience guiding students who joined KavrakiLab and worked on projects in the field of structural computational biology and applied machine learning.
\end{minipage}
}
\\ \\
\multicolumn{2}{@{}X@{}}{
\begin{minipage}[t]{\linewidth}
\begin{itemize}[nosep,after=\strut, leftmargin=1em, itemsep=3pt]
        \item[--] Jaila Lewis (University of Houston) - Summer 2022
        \item[--] Aleksandar Gavric (University of Belgrade) - Summer 2021
        \item[--] Davyd Fridman (Rice University) - Summer 2021
        \item[--] Nonso Chukwurah (Rice University) - Spring 2021
    \end{itemize}
\end{minipage}
}
\end{tabularx}

\begin{tabularx}{\linewidth}{ @{}l r@{} }

\textbf{Student demonstrator at University of Belgrade} & \hfill 2018 \\
\textit{School of Electrical Engineering, University of Belgrade} & \\

\multicolumn{2}{@{}X@{}}{
\begin{minipage}[t]{\linewidth}
Tutoring students in the Software Design course, presenting the StarUML tool and grading the
works of students. \end{minipage}
}
\end{tabularx}


%----------------------------------------------------------------------------------------
%	PUBLICATIONS
%----------------------------------------------------------------------------------------
\section{Publications}
\begin{refsection}[citations.bib]
\nocite{*}
\printbibliography[heading=none]
\end{refsection}


%----------------------------------------------------------------------------------------
%	PROJECTS
%----------------------------------------------------------------------------------------

\section{Open-source projects}

\begin{tabularx}{\linewidth}{ @{}l r@{} }
 \href{https://github.com/anon528/supreme-couscous}{\faGithub \textbf{EnGens}} &\\[3.75pt]
\multicolumn{2}{@{}X@{}}{}  
\begin{minipage}[t]{\linewidth}
EnGens is a computational framework for generation and analysis of representative protein conformational ensembles. EnGens performs dimensionality reduction and clustering of the structural datasets to identify representatives from each cluster into a subset of a representative ensemble that can be used for downstream tasks. EnGens is provided as a Python package wrapped in a Docker image and accopanied by multiple interacting Jupyter Notebooks.    
\end{minipage}
\end{tabularx}

\begin{tabularx}{\linewidth}{ @{}l r@{} }
 \href{https://github.com/KavrakiLab/SARS-Arena}{\faGithub \textbf{SARS-Arena}} &\\[3.75pt]
\multicolumn{2}{@{}X@{}}{}  
\begin{minipage}[t]{\linewidth}
This work emerged as our response to the SARS-Cov-2 pandemic. Leveraging the resources we previously developed, we tuned our antigen prediction pipelines to fit the purpose of SARS-Cov-2 related antigen discovery. Documentation and installation instructions are provided in the public github repository. For easy access, a public Docker image is built with all requirements installed.      
\end{minipage}
\end{tabularx}

\begin{tabularx}{\linewidth}{ @{}l r@{} }
 \href{https://github.com/KavrakiLab/3pHLA-score}{\faGithub \textbf{3pHLA-score}} &\\[3.75pt]
\multicolumn{2}{@{}X@{}}{}
\begin{minipage}[t]{\linewidth}
In this work we explored the use of ML models for training a system-specific scoring function for the purpose of scoring the binding energy of peptide ligands to HLA protein receptors. The main contribution includes the novel per-peptide-position training protocol as a proof of concept. The deliverables include a Python package, documentation and instructions within the Github repository as well as a Docker image.
\end{minipage}
\end{tabularx}

\begin{tabularx}{\linewidth}{ @{}l r@{} }
 \href{https://github.com/KavrakiLab/HLA-Arena}{\faGithub \textbf{HLA-Arena}} &\\[3.75pt]
\multicolumn{2}{@{}X@{}}{} 
\begin{minipage}[t]{\linewidth}
HLA-Arena provides an interactive pipeline for structural computational modeling and analysis of the human leukocite antigen (HLA) protein and its interaction with the peptide ligands (antigens). The work is packaged into a Docker image and the workflows are provided through a set of Jupyter Notebooks.
\end{minipage}
\end{tabularx}


%----------------------------------------------------------------------------------------
%	AWARDS
%----------------------------------------------------------------------------------------

\section{Awards and Scholarships}

\begin{tabularx}{\linewidth}{ @{}l r@{} }
\textbf{ITN travel award} & \hfill Sep 2022 \\
\textit{\href{https://itcr2022.org/}{ITCR 2022}} & \\[3.75pt]
\multicolumn{2}{@{}X@{}}{Travel award to visit the NCI Informatics Technology for Cancer Research (ITCR) conference and participate in a training workshop. }
\end{tabularx}

\begin{tabularx}{\linewidth}{ @{}l r@{} }
\textbf{Rice Datathon 2022 - Second Place in the Bill.com Challenge} & \hfill Jan 2022 \\
\textit{\href{https://d2k.rice.edu/rice-datathon-2022}{Rice Datathon 2022}} & \\[3.75pt]
\multicolumn{2}{@{}X@{}}{For our work titled \href{https://devpost.com/software/insight-into-connection}{\textit{Insight into Connection}} on link prediction task with a graph neural network approach.}
\end{tabularx}

\begin{tabularx}{\linewidth}{ @{}l r@{} }
\textbf{Poster Presentation Session: Potential Impact Award} & \hfill Oct 2020 \\
\textit{2020 Ken Kennedy Institute Data Science Conference} & \\[3.75pt]
\multicolumn{2}{@{}X@{}}{For the poster titled: "Combining Structure and Sequence Data to Predict Peptide-HLA Binding Affinity"}
\end{tabularx}

\begin{tabularx}{\linewidth}{ @{}l r@{} }
\textbf{"Dositeja" scholarship for studying abroad} & \hfill 2019-2021 \\
\textit{Ministry of Youth and Sport, Republic of Serbia} & \\[3.75pt]
\multicolumn{2}{@{}X@{}}{"Dositeja" is a scholarship awarded by the Fund for Young Talents of the Republic of Serbia to talented and successful students. In addition to the strength of the student’s academic record, criteria is based on the specific institution of further study.}
\end{tabularx}

%----------------------------------------------------------------------------------------
%	WORK EXPERIENCE
%----------------------------------------------------------------------------------------

\section{Work experience}

\begin{tabularx}{\linewidth}{ @{}l r@{} }
\textbf{Junior full stack web developer} & \hfill Nov 2018 - Jun 2019 \\
\textit{\href{https://www.efront.com/}{eFront}, Belgrade} & \\[3.75pt]
\multicolumn{2}{@{}X@{}}{Full-stack web development experience with React, Java frameworks}
\end{tabularx}

\begin{tabularx}{\linewidth}{ @{}l r@{} }
\textbf{Web development Internship} & \hfill Feb - Apr 2018 \\
\textit{\href{https://vicert.com/}{Pamet.doo}, Belgrade} & \\[3.75pt]
\multicolumn{2}{@{}X@{}}{Full-stack web development experience with Angular, React, Java frameworks}
\end{tabularx}

\begin{tabularx}{\linewidth}{ @{}l r@{} }
\textbf{Student reviewer} & \hfill 2017 \\
\textit{\href{http://www.kapk.org/en/home/}{Commission for accreditation and quality assurance in higher education}, Serbia } & \\[3.75pt]
\multicolumn{2}{@{}X@{}}{Student representative writing reports and conducting site-visits and interviews of higher education institutions}
\end{tabularx}

\begin{tabularx}{\linewidth}{ @{}l r@{} }
\textbf{Programming internship} & \hfill Oct - Dec 2016 \\
\textit{\href{https://www.mikroe.com/}{MikroElektronika}, Belgrade } & \\[3.75pt]
\multicolumn{2}{@{}X@{}}{Embedded systems programming in C, C++}
\end{tabularx}


%----------------------------------------------------------------------------------------
%	SKILLS
%----------------------------------------------------------------------------------------
\section{Skills and hobbies}
\begin{tabularx}{\linewidth}{@{}l X@{}}
Programming skills &  \normalsize{Python, R, C, C++, Java}\\
Operating systems &  \normalsize{Linux, Windows}\\
Communication skill & \normalsize{good communication skills gained through my experience with collaborative interdisciplinary research as well as past experiences as a television presenter and organizer, and in a youth theater group} \\
Organizational skills  &  \normalsize{good communication skills gained through the experience in mentorship, teaching and collaborative projects}\\  
Hobbies &  \normalsize{guitar, drama, yoga, \href{http://blablablatruc.blogspot.com/}{creative writing}}\\
\end{tabularx}

\vfill
\center{\footnotesize Last updated: \today}

\end{document}
